Waterflooding is an enhanced oil recovery technique widely employed in the Oil \& Gas industry. Its main purpose is to increase overall oil recovery factors from a reservoir asset by injecting water at injection wells located in different reservoir zones~\citep{}. The importance of waterflooding increases as a field becomes mature. Mature fields are characterized by a decline in oil production due to reserve depletion, which causes a decrease in reservoir pressure. Injecting of water through waterflooding enables one to maintain reservoir pressure while causing a sweeping effect over reminiscent oil towards  production wells~\citep{}. In offshore facilities, waterflooding can be implemented by either injecting ocean water or produced water. In the Norwegian Continental Shelf, a shift towards re-injecting produced water in the reservoir has been observed~\citep{}. This trend follows the zero discharge policy recommended by \citep{} to reduce marine pollution caused by offshore oil and gas production. Proper operation of a produced water re-injection facility is a challenging task as there is an increasing need to improve the facility economic and environmental performance. 

\par A MINLP model has been developed in \citep{Zhou2019} for the optimal operation of waterflooding strategy. However, in this work it has been considered the presence of only fixed-speed pumps (FSP).


\par Model predictive control (MPC) is a te


According to~\citep{}, 

\par In the traditional two-layered approach, the model employed in the real-time optimization layer should represent the process with a higher fidelity than the model in the lower layer. Contrary to the traditional approach, we showcase that the model at the MINLP layer does not necessarily need to present a higher fidelity than the one in the EMPC layer. In fact, the model at the MINLP layer should be rich enough to enable one to find the optimal steady-state integer variables. Once this is done, the EMPC layer becomes the sole responsible for keeping the process at its operational optimum.

\par The work is structured as follows:
\begin{itemize}
    \item The virtual plant model as well as the main objectives are described in this section.
    \item The hierarchical control structure is presented in this section.
    \item The MINLP srategy is described in this section.
    \item The EMPC strategy is described in this section.
\end{itemize}